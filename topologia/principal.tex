\chapter{Topologia}
Estas anotações estão fortemente relacionadas ao curso de topologia lecionado pelo professor Athanasios Thanos Tsouanas. O curso está sendo ofertado em uma turma do componente IMD0109 - Tópicos Epeciais em Tecnologia da Informação X no semestre 2019.1 na UFRN.

As notas estão bem desorganizadas, como se pode perceber. Ainda organizarei tudo direitinho quando puder.

\section{Minhas anotações}

Considere $X$ não vazio e $\caligrafico{N}$ uma família de subconjuntos de $X$. 

\begin{definition}[Sistema de Vizinhança]
	Chamamos $\caligrafico{N}$ de um sistema de vizinhança no $X$ se, e somente se:
\begin{enumerate}[a)]
\item
	Todo elemento $p \no X$ habita em pelo menos uma vizinhança $N \no \caligrafico{N}$. Mais formalmente, para todo $p \no X$, existe $N \in \caligrafico{N}$ tal que $p \no N$.

\item
	Se algum $p$ habita em duas vizinhanças $N_1$ e $N_2$, então deve existir uma terceira vizinhança $N \contido N_1 \inter N_2$ em que o $p$ habita. Mais formalmente, se $p \no N_1 \cap N_2$, então existe $N \contido N_1 \inter N_2$ tal que $p \no N$.
\end{enumerate}
\end{definition}

\begin{definition}[Espaço Topológico]
	Suponha que $\caligrafico{N}$ é um sistema de vizinhança no $X$, então chamamos a estrutura $\caligrafico{T} = (X ; \caligrafico{N})$ de espaço topológico.
\end{definition}


\begin{definition}[Conjunto Aberto]
	Chamamos um subconjunto $O$ de $X$ de aberto se, e somente se, todos os seus elementos habitam em uma vizinhança. Mais formalmente:
	\begin{center}
		$O$ é aberto \sse para todo $p \no O$, existe $N \no \caligrafico{N}$ tal que $p \no N$.
	\end{center}
\end{definition}

\begin{fact}
	Toda vizinhança é um conjunto aberto.
\end{fact}

\begin{proof}
	Considere uma vizinhança $N \no \caligrafico{N}$. Tome um arbitrário $p \no N$. Agora basta mostrar que o $p$ habita em alguma vizinhança. Ora, como o $N$ já é uma vizinhança e já temos que $p \no N$, então $N$ é aberto.
\end{proof}

\begin{definition}
	A topologia gerada por \caligrafico{N} é a família \caligrafico{O} de todos os seus conjuntos abertos.
\end{definition}