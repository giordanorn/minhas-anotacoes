\chapter{Topologia}
Estas anotações estão fortemente relacionadas ao curso de topologia lecionado pelo professor Athanasios Thanos Tsouanas. O curso está sendo ofertado em uma turma do componente IMD0109 - Tópicos Epeciais em Tecnologia da Informação X no semestre 2019.1 na UFRN.

As notas estão bem desorganizadas, como se pode perceber. Ainda organizarei tudo direitinho quando puder.

\section{Minhas anotações}
Para começarmos a estudar os conceitos de topologia, precisamos de um conjunto $X$ não-vazio, e uma família $\caligrafico{N}$ formada por subconjuntos de $X$. A seguir definiremos algumas notações:
\begin{itemize}
	\item Chamamos o conjunto $X$ de universo.
	\item Chamamos os elementos $p \no X$ de pontos.
	\item Chamamos os elementos $N \no \caligrafico{N}$ de vizinhanças.
	\item Se acontece que $p \no X$ pertence à alguma vizinhança $N \no \caligrafico{N}$, chamamos o $N$ de uma vizinhança do $p$. Com isso, podemos definir o predicado abaixo:
	\begin{center}
		$N$ é uma vizinhança do $p$ \sse $p \no N$.
	\end{center}
\end{itemize}

Quase todo mundo recebeu um nome, ficou faltando apenas a família \caligrafico{N}. Ela também receberá um nome, veremos o que ela precisa satisfazer para receber esse nome na definição a seguir.

\begin{definition}[Sistema de vizinhança]
	\label{def:sistema-de-vizinhanca}
	Chamamos $\caligrafico{N}$ de um sistema de vizinhança no $X$ se, e somente se:
	\begin{description}
		\item[(NS1)]
		\label{def:sistema-de-vizinhanca1}
		Todo ponto $p$ habita em pelo menos uma vizinhança $N$. Em fórmulas:
		\begin{center}
			($\paratodo p \no X$)%
			($\existe N \no \caligrafico{N}$)%
			[ $p \no N$ ]
		\end{center}

		\item[(NS2)]
		\label{def:sistema-de-vizinhanca2}
		Quaisquer duas vizinhanças $N_1$, $N_2$ do $p$ ao serem intersetadas devem conter uma vizinhança $N$ do $p$, isto é, $N \contido N_1 \inter N_2$. Em fórmulas:
		\begin{center}
			($\paratodo N_1, N_2 \no \caligrafico{N}$)%
			($\paratodo p \no N_1 \inter N_2$)%
			($\existe N \no \caligrafico{N}$)%
			[ $p \no N$ \& $N \contido N_1 \inter N_2$ ]
		\end{center}
	\end{description}
\end{definition}

\begin{definition}[Espaço topológico]
	Se $\caligrafico{N}$ é um sistema de vizinhanças em $X$, então chamamos a estrutura $(X;\caligrafico{N})$ de um espaço topológico.
\end{definition}

\begin{definition}[Aberto]
	\label{def:aberto}
	Chamamos um conjunto $O \contido X$ de aberto em \caligrafico{N}, ou, simplesmente de aberto (quando o sistema de vizinhanças estiver implícito pelo contexto) se, e somente se, todos os seus elementos possuem alguma vizinhança contida no $O$. Mais formalmente:
	\begin{center}
		$O$ é aberto em \caligrafico{N} \sse
			($\paratodo p \no O$)%
			($\existe N \no \caligrafico{N}$)%
			[ $p \no N$ \& $N \contido O$ ].
	\end{center}
\end{definition}

\begin{fact}
	\label{vizinhanca-aberto}
	Toda vizinhança é um conjunto aberto.
\end{fact}

\begin{proof}
	Considere uma vizinhança $N \no \caligrafico{N}$. Tome um arbitrário $p \no N$. Agora basta mostrar que o $p$ habita em alguma vizinhança contida no $N$. Ora, como $p \no N$, $N$ é uma vizinhança e $N \contido N$, então $N$ é aberto.
\end{proof}

\begin{question}
	\label{universo-aberto}
	\label{vazio-aberto}
	O que dizer sobre os $X$ e $\vazio$? São abertos?
\end{question}

\begin{answer}
	Sim, ambos são abertos. Primeiramente mostremos que o $X$ é aberto:
	\begin{proof}
		Precisamos mostrar que todo elemento do $X$ habita em pelo menos uma vizinhança $N \contido X$. Ora, isso é o que o próprio item~\ref{def:sistema-de-vizinhanca1} afirma. Sendo assim, o $X$ é aberto.
	\end{proof}
	Agora, demonstremos que o $\vazio$ é aberto.
	\begin{proof}
		Precisamos mostrar que todo elemento do $\vazio$ habita em pelo menos uma vizinhança $N \contido X$. O que já é verdade por vacuidade. Sendo assim o $\vazio$ é aberto.
	\end{proof}
\end{answer}

\begin{definition}[Topologia]
	\label{def:topologia}
	A topologia gerada por $\caligrafico{N}$ é a família $\topologia{N}$ de todos os seus conjuntos abertos.
	\[
		\topologia{N} = \conjunto{ O \contido X \tq \text{$O$ é aberto em \caligrafico{N}}}
	\]
\end{definition}

\begin{definition}[Vizinhança punturada]
	\label{def:vizinhanca-punturada}
	Sejam $A \contido X$ e $p \no X$. Definimos o predicado abaixo:
	\begin{center}
		$A$ é uma vizinhança punturada do $p$ $\sse$ $p \fora\no A$ \& $A \uniao \conjunto{p} \no \caligrafico{N}$.
	\end{center}
	\begin{example}
		Dados um ponto $p \no X$ e uma vizinhança $N \no \caligrafico{N}$ do $p$, isto é, $p \no N$, denotamos o conjunto $\punturado{N}{p} := N \menos \conjunto{p}$. Então $\punturado{N}{p}$ é uma vizinhança punturada do $p$. Observe que $p \fora\no \punturado{N}{p}$ e $\punturado{N}{p} \uniao \conjunto{p} = N \no \caligrafico{N}$.
		Sendo assim, $\punturado{N}{p}$ é uma vizinhança punturada do $p$.
	\end{example}
\end{definition}

\begin{definition}[Limite]
	\label{def:limit-point}
	Dado $A \contido X$, dizemos que $l \no X$ é um limite de $A$ ou $A \limite l$ se, e somente se, toda vizinhança punturada do $l$ interseta o $A$. Mais formalmente:
	\begin{center}
		$A \limite l$ \sse
			($\paratodo N \no \caligrafico{N}$)%
			[ $l \no N \implica \punturado{N}{l} \inter A \diferente \vazio$ ]
	\end{center}
	A idéia é que você não consegue isolar o $l$ do $A$ com alguma vizinhança.
\end{definition}

\begin{definition}[Conjunto derivado]
	\label{def:derivado}
	Chamamos o conjunto de todos os limites de $A$ de conjunto derivado de $A$ e o denotamos como $\derivado{A}$. Isto é:
	\[
		\derivado{A} = \conjunto{ l \no X \tq A \limite l}
	\]
\end{definition}

\begin{definition}[Fechado]
	\label{def:fechado}
	Seja $A \contido X$, chamamos o $A$ de fechado se, e somente se, o $A$ possui todos os seus limites. Em fórmulas:
	\begin{center}
		($\paratodo l \no X$)[ $A \limite l \implica l \no A$ ].
	\end{center}
	O que também pode ser visto da seguinte forma:
	\[
		\derivado{A} \contido A.
	\]
\end{definition}

\begin{question}
	\label{universo-fechado}
	\label{vazio-fechado}
	O que dizer sobre os $X$ e $\vazio$? São fechados?
\end{question}

\begin{answer}
	Ambos são fechados. Primeiramente, demonstremos que X é fechado:
	\begin{proof}
		Precisamos demonstrar que $X$ possui todos os seus limites. Ora, como $X$ é o nosso universo, já temos que qualquer possível limite de $X$ já habita no $X$. Sendo assim, o $X$ é fechado.
	\end{proof}
	Agora, demonstremos que o $\vazio$ é fechado:
	\begin{proof}
		Mostrarei que o $\vazio$ não possui limites. Para isso, seja $p \no X$. Mostrarei que $\vazio \nao\limite p$, isto é, $\derivado{\vazio} = \vazio$. Note que, existe $N \no \caligrafico{N}$ tal que $p \no N$. E, como $\punturado{N}{p} \inter \vazio = \vazio$, temos uma vizinhança punturada do $p$ que não partilha  elementos com o $\vazio$. Portanto $\vazio \nao\limite l$ e, com isso, concluímos que o $\vazio$ é fechado.
	\end{proof}
\end{answer}

\begin{question}
	Um conjunto unitário é fechado?
\end{question}

\begin{answer}
	Depende. Há casos que é fechado e casos que não é fechado. Mostremos um caso que é fechado:
	\begin{proof}	
		Basta tomar seu universo como sendo $X = \conjunto{a}$. Com isso, teremos que o $X$ é unitário. E, como já demonstramos em \ref{universo-fechado} que todo universo é fechado, já terminamos. 
	\end{proof}
	Agora, mostremos um caso que não é fechado:
	\begin{proof}
		Basta tomar seu universo como sendo $X = \conjunto{a,l}$, e o sistema de vizinhanças $\caligrafico{N} = \conjunto{X}$. Com isso, teremos que o conjunto unitário $A = \conjunto{a} \contido X$ não é fechado. Observe que $A \limite l$, pois a única vizinhança punturada do $l$, que é o $\punturado{X}{l} = \conjunto{a}$, partilha elementos de $A$, isto é, $\punturado{X}{l} \inter A \neq \vazio$ graças ao $a$. Porém, $l \fora\no A$. Portanto, o $A$ não é fechado mesmo sendo unitário.
	\end{proof}
\end{answer}

\begin{question}
	Se $p$ é um limite de $A \inter B$, então $p$ é um limite de $A$ e $p$ é um limite de $B$?
\end{question}

\begin{question}
	Se $p$ é um limite de $A \uniao B$, então $p$ é um limite de $A$ ou $p$ é um limite de $B$?
\end{question}

\begin{definition}[Fecho de um conjunto]
	\label{def:fecho}
	Dado um conjunto $A \contido X$, chamamos de fecho de $A$ o conjunto formado de elementos de $A$ com todos os seus limites. O denotamos como $\fechado{A}$. Isto é:
	\[
		\fechado{A} = A \uniao \derivado{A}.
	\]
\end{definition}

\begin{theorem}[Fechos são fechados]
	\label{thm:fecho-fechado}
	Para todo conjunto $A \contido X$, temos que o $\fechado{A}$ é fechado.
\end{theorem}

\begin{proof}
	A fazer...
\end{proof}

\begin{definition}[Ponto interior]
	\label{def:ponto-interior}
	Seja um conjunto $A \contido X$ e um ponto $p \no A$. Chamamos $p$ de um ponto interior de $A$ se, e somente se, o $p$ possui uma vizinhança contida no $A$:
	\begin{center}
		$p$ é ponto interior de $A$ \sse
			($\existe N \no \caligrafico{N}$)%
			[ $p \no N$ \& $N \contido A$ ].
	\end{center}
\end{definition}

\begin{definition}[Interior de um conjunto]
	\label{def:interior}
	Dado um conjunto $A \contido X$, chamamos de interior de $A$ o conjunto formado por todos os pontos interiores de $A$. O denotamos como $\interior{A}$. Isto é:
	\[
		\interior{A} = \conjunto{ i \no A \tq \text{$i$ é ponto interior de $A$} }
	\]
\end{definition}

\begin{fact}
	\label{fact:interior-contido}
	O interior de um conjunto está contido no conjunto, isto é, $\interior{A} \contido A$.
\end{fact}

\begin{theorem}[Interiores são abertos]
	\label{thm:interior-aberto}
	Para todo conjunto $A \contido X$, temos que o $\interior{A}$ é aberto.
\end{theorem}

\begin{proof}
	Precisamos mostrar que para todo interior $i \no \interior{A}$ existe uma vizinhança $N \contido \interior{A}$ do $i$, ou seja, $i \no N$. 
	
	Sendo assim, seja $i \no \interior{A}$, isto é, $i$ é um ponto interior de $A$. Então, existe uma vizinhança $N \contido A$ tal que $i \no N$. Observe que o $N$ é a vizinhança do $i$ que testemunha $N \contido \interior{A}$. Basta ver que não existem elementos do $N$ fora do $\interior{A}$, isto é, $N \contido \interior{A}$.
	
	De fato, observe que se $p \no N$, então $p$ é um ponto interior de $A$ pois $N \contido A$, ou seja $p \no \interior{A}$. Então, $N \contido \interior{A}$. Sendo assim, como achamos uma vizinhança $N$ do $i$ que satisfaz $N \contido \interior{A}$, temos que o $\interior{A}$ é aberto, como queríamos demonstrar.
\end{proof}

\begin{corollary}
	Um conjunto $A$ é aberto se, e somente se $\interior{A} = A$.
\end{corollary}

\begin{question}
	Se $p$ é um ponto interior de $A \inter B$, então $p$ é um ponto interior de $A$ e $p$ é um ponto interior de $B$?
\end{question}

\begin{question}
	Se $p$ é um ponto interior de $A \uniao B$, então $p$ é um ponto interior de $A$ ou $p$ é um ponto interior de $B$?
\end{question}

\begin{theorem}
	\label{thm:interior-fecho}
	O complemento do interior é o fecho do complemento. Isto é, para qualquer $A \contido X$, vale:
	\[
		\complemento{\interior{A}} = \derivado{\complemento{A}}
	\]
\end{theorem}

\begin{proof}
	A fazer...
\end{proof}

\begin{definition}[Subconjunto aberto]
	\label{def:subconjunto-aberto}
	Um conjunto $Y$ é aberto no $A$ se, e somente se, existe um conjunto aberto $O$ tal que $Y = A \inter O$ . Também chamamos o $Y$ de um subconjunto aberto do $A$.
\end{definition}

\begin{question}
	Todo subconjunto aberto de um conjunto é aberto?
\end{question}

\begin{definition}[Base]
	\label{def:base}
	Uma família $\caligrafico{B}$ de subconjuntos de $X$ é uma base para $\topologia{N}$ se, e somente se:
	\begin{enumerate}[i)]
		\item Todo membro do $\caligrafico{B}$ é aberto. Isto é, $\caligrafico{B} \contido \topologia{N}$.
		\item Cada abeto $O \no \topologia{N}$ pode ser escrito como uma união de elementos da base $\caligrafico{B}$. Mais formalmente, para todo $O \no \topologia{N}$ temos $O = \Uniao \familia{B}$ sendo $B_i \no \caligrafico{B}$ para alguma família $\familia{B} \contido \caligrafico{B}$.
	\end{enumerate}
\end{definition}

\begin{fact}
	$\caligrafico{N}$ é uma base para $\topologia{N}$.
\end{fact}

\begin{theorem}
	Se $\caligrafico{B}$ é uma base para $\topologia{N}$, então $\caligrafico{B}$ é um sistema de vizinhança no $X$ gerando a mesma topologia $\topologia{B} = \topologia{N}$.
\end{theorem}

\begin{proof}
	Primeiramente mostremos que $\caligrafico{B}$ é um sistema de vizinhança do $X$.
	\begin{itemize}
		\item[(NS1)] Seja $p \no X$. Como $X$ é aberto, então $X = \Uniao \familia{B}$ pois $\caligrafico{B}$ é uma base para $\topologia{N}$. Sendo assim, $p \no B_i$ para algum $i \no \caligrafico{I}$. Portanto, $p \no B_i \no \caligrafico{B}$ como queríamos demonstrar.
		
		\item[(NS2)] Sejam $B_1$, $B_2 \no \caligrafico{B}$ tais que $p \no B_1 \inter B_2$. Como $B_1$ é aberto, então existe $N_1 \contido B_1$ tal que $p \no N_1$. E, como $B_2$ é aberto, temos que existe $N_2 \contido B_2$ talque $p \no N_2$. Sendo assim, temos $p \no N_1 \inter N_2$. Então, existe $N \contido N_1 \inter N_2$ tal que $p \no N$. Sabemos que $N$ é aberto, então $N = \Uniao \familia{B}$ para alguma $\familia{B} \contido \caligrafico{B}$, pois $\caligrafico{B}$ é uma base para $\topologia{N}$. Sendo assim, $p \no B_i$ para algum $i \no \caligrafico{I}$. Com isso, temos $p \no B_i \contido \Uniao \familia{B} = N \contido B_1 \inter B_2$, como queríamos demonstrar.
	\end{itemize}
\end{proof}
