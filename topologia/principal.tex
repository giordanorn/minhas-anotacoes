\chapter{Topologia}
Estas anotações estão fortemente relacionadas ao curso de topologia lecionado pelo professor Athanasios Thanos Tsouanas. O curso está sendo ofertado em uma turma do componente IMD0109 - Tópicos Epeciais em Tecnologia da Informação X no semestre 2019.1 na UFRN.

As notas estão bem desorganizadas, como se pode perceber. Ainda organizarei tudo direitinho quando puder.

\section{Minhas anotações}

Considere $X$ não vazio e $\caligrafico{N}$ uma família de subconjuntos de $X$. 

\begin{definition}[Sistema de vizinhança --- Neighbourhood system]
	\label{def:sistema-de-vizinhanca}
	Chamamos $\caligrafico{N}$ de um sistema de vizinhança no $X$ se, e somente se:
	\begin{enumerate}[a)]
	\item
		\label{def:sistema-de-vizinhanca1}
		Todo elemento $p \no X$ habita em pelo menos uma vizinhança $N \no \caligrafico{N}$. Mais formalmente, para todo $p \no X$, existe $N \in \caligrafico{N}$ tal que $p \no N$.

	\item
		\label{def:sistema-de-vizinhanca2}
		Se algum $p$ habita em duas vizinhanças $N_1$ e $N_2$, então deve existir uma terceira vizinhança $N \contido N_1 \inter N_2$ em que o $p$ habita. Mais formalmente, se $p \no N_1 \cap N_2$, então existe $N \contido N_1 \inter N_2$ tal que $p \no N$.
\end{enumerate}
\end{definition}

\begin{definition}[Espaço topológico --- Topological space]
	Suponha que $\caligrafico{N}$ é um sistema de vizinhança no $X$, então chamamos a estrutura $\caligrafico{T} = (X ; \caligrafico{N})$ de espaço topológico.
\end{definition}


\begin{definition}[Aberto --- Open]
	Chamamos um subconjunto $O$ de $X$ de aberto se, e somente se, todos os seus elementos habitam em pelo menos uma vizinhança contida no $O$. Mais formalmente:
	\begin{center}
		$O$ é aberto \sse para todo $p \no O$, existe $N \no \caligrafico{N}$ tal que $p \no N$ e $N \contido O$.
	\end{center}
\end{definition}

\begin{fact}
	Toda vizinhança é um conjunto aberto.
\end{fact}

\begin{proof}
	Considere uma vizinhança $N \no \caligrafico{N}$. Tome um arbitrário $p \no N$. Agora basta mostrar que o $p$ habita em alguma vizinhança contido no $N$. Ora, como $N \contido N$, $N$ já é uma vizinhança e já temos que $p \no N$, então $N$ é aberto.
\end{proof}

\begin{definition}
	A topologia gerada por \caligrafico{N} é a família \caligrafico{O} de todos os seus conjuntos abertos.
\end{definition}

\begin{definition}[Limit point]
	Dado $A \contido X$, dizemos que $p$ é um \textit{limit point} de $A$ se, e somente se, em toda vizinhança que o $p$ habita há pelo menos um elemento diferente do $p$ que também habita nela. Mais formalmente:
	\begin{center}
		$p$ é limit point de $A$ \sse todo $N$ que o $p$ habita tem um $q \no N$ tal que $q \neq p$. 
	\end{center}
\end{definition}

\begin{definition}
	Chamamos de $A^\linha$ o conjunto de todos os limit points de $A$.
\end{definition}

\begin{definition}[Fechado --- Closed]
	Seja $A \contido X$, chamamos o $A$ de fechado se, e somente se, $A$ possui todos os seus limit points. Em outras palavras, $A^\linha \contido A$.
\end{definition}

\begin{question}
	\label{universo-fechado}
	\label{vazio-fechado}
	O que dizer sobre os $X$ e $\vazio$? São fechados?
\end{question}

\begin{answer}
	Ambos são fechados. Primeiramente, demonstremos que X é fechado.
	\begin{proof}
		Precisamos demonstrar que $X$ possui todos os seus limit points. Ora, como $X$ é o nosso universo, já temos que qualquer possível limit point de $X$ já habita no $X$. Sendo assim $X^\linha \contido X$.
	\end{proof}

	Agora, demonstremos que o $\vazio$ é fechado.
	\begin{proof}
		Considere $p \no X$. Basta mostrar que $p$ não é limit point do $\vazio$. Note que, existe $N \no \caligrafico{N}$ tal que $p \no N$ pelo item~\ref{def:sistema-de-vizinhanca1}. E, como $N \inter \vazio = \vazio$, temos que os $N$ e $\vazio$ não partilham elementos diferentes do $p$. Portanto $\vazio^\linha \contido \vazio$.
	\end{proof}
\end{answer}

\begin{question}
	Um conjunto unitário é fechado?
\end{question}

\begin{answer}
	Depende. Há casos que é fechado e casos que não é fechado.
	
	Mostremos um caso que é fechado.
	\begin{proof}	
		Basta tomar seu universo como sendo $X = \conjunto{a}$. Com isso, teremos que o $X$ é unitário. E, como já demonstramos em \ref{universo-fechado} que todo universo é fechado, já terminamos. 
	\end{proof}

	Agora, mostremos um caso que não é fechado.
	\begin{proof}
		Basta tomar seu universo como sendo $X = \conjunto{a,l}$, e o sistema de vizinhanças $\caligrafico{N} = \conjunto{X}$. Com isso, teremos que o conjunto unitário $A = \conjunto{a} \contido X$ não é fechado. Observe que $l$ é um limit point de $A$, pois a única vizinhança do $l$, que é o próprio $X$, partilha elementos de $A$ que não são o $l$, no caso o $a$. Porém, $b \fora\no A$. Portanto, o $A$ não é fechado mesmo sendo unitário.
	\end{proof}
\end{answer}